\chapter{Background}\label{ch:A}
\section{Literature review} \label{litrev}

In our world of non-stop informational flow and worldwide internet access, we have the means to access a vast array of information. Yet, students often encounter difficulties in various courses. In the articles "Massive Open Online Courses (MOOCs) \cite{massiveopenonlinecourses}: A Primer for University and College Board Members" and "SWATShare \cite{swatShare}– A web platform for collaborative research and education through online sharing, simulation, and visualization of SWAT models," we were introduced to projects aimed at assisting individuals with educational goals. 
\par
\vspace{0.5cm}
The most inspiring aspect of the first article \cite{massiveopenonlinecourses} was the project's similarity to on-campus courses, delivered synchronously on a defined schedule—usually on a weekly calendar basis—and offered for free. This prompted us to consider embedding similar features in our platform. We envisioned a platform where every student could share their skills, with options for free access, payment, or exchange for other skills or courses offered by fellow students. This approach would provide a more personalized learning experience, similar to receiving guidance from peers who have experienced similar challenges, thereby enhancing the effectiveness of education.
\par
\vspace{0.5cm}
In the second article \cite{swatShare}, we found it intriguing that the project emphasized the importance of collaboration among researchers and educators. This aligns with our main goal of facilitating easier, more interesting, and more useful communication among students for their educational benefit.
\par
\vspace{0.5cm}
Since we are aiming to embed crowdfunding \cite{crowdfunding} functionality into our project, we found the article "Crowdfunding: Why People Are Motivated to Post and Fund Projects on Crowdfunding Platforms" quite informative. This article provided us with insights into how crowdfunding works from both the creator and funder perspectives. It highlighted how social interactions play a crucial role in creating a feeling of belonging to a community that shares common interests and values, especially among those who fund projects. Additionally, the article discussed the psychological aspects of crowdfunding, highlighting factors such as sympathy, empathy, guilt, happiness, and identity, which influence individuals' motivations for investing money into projects. It also addressed how the framing of a funding request can impact the donation amount and the potential limitations we might encounter in the process.

\section{International and local cases analysis} \label{intandloc}

In the process of brainstorming our project's problem statement, we considered already existing projects created for students to develop a competitive project. We decided to examine six different cases for the three main functionalities of our app: skill sharing, Question Answer, and crowdfunding, with 1 local and 1 international existing solutions for each functionality.

\begin{enumerate}
    \item First, let’s consider skill sharing functionality in next platforms:

     \textit{SkillShare} is a platform designed for sharing skills, enabling users to either purchase or upload their courses. It is a globally recognized platform that inspired us, but we observed that despite feedback, many course-selling platforms like skillshare often lack effective communication and support. This led us to explore local application \textit{Olx}. Olx \cite{olx} features a service-sharing functionality that enhances user interaction significantly. However, its major drawback lies in the lack of security measures, resulting in numerous scams and a suspicious reputation. In response, we conceived the idea of launching a platform accessible only to authenticated users, exclusively students. Through thorough analysis, we identified the strengths of both platforms and endeavoured to address their shortcomings.

     \item Next, let’s consider Question Answer functionality in next platforms:

     Our team drew inspiration from the platform \textit{Stack Overflow} \cite{stackoverlow}, which has provided answers to our questions on numerous occasions. With its wide user base, responses to questions are swift, and the variety of people who contribute brings different viewpoints, which is a significant benefit. However, we identified a limitation: Stack Overflow \cite{stackoverlow} created exclusively for developers, whereas we aspire to create a platform where students from various fields can engage in discussions and seek solutions to their questions. In our quest for alternatives, we explored the Russian platform \textit{Znanija.com} \cite{znanija}, tailored for school students and covering a broader range of subjects. Nevertheless, a significant drawback became apparent: the platform's limited user base and language barrier, resulting in fewer responses to submitted questions and issues. Another common disadvantage in both of these platforms is also authentication, allowing any user to provide answers, consequently, there may be no way to follow up with the user who previously answered. Therefore, in our application, we aim to enable authenticated students to engage in discussions about topics of interest and establish connections with users who provide answers, similar to interactions on social media platforms.

  \item Finally, let’s consider crowdfunding functionality in next platforms:

We came across an interesting student-centered crowdfunding platform called \textit{StudentBackr}, which is designed to help students cover various fees and finance projects related to student life. However, we discovered that it operates exclusively in European countries.So we looked into local \textit{Start-time.kz} \cite{starttime} platform, which, although not specifically tailored for students, still allows investments in various areas of interest, which is excellent, but the platform's biggest flaw lies in its consistency. Platform is not actually user friendly and we could not get acquainted with terms and conditions or any policy. So we considered both strengths and weaknesses of these projects to develop competitive application. Our goal is to launch a user-centered crowdfunding platform that prioritizes usefulness and meets all requirements of such a significant project and will bring up projects with potential.
\end{enumerate}